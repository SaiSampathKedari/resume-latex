%%%%%%%%%%%%%%%%%
% This is an sample CV template created using altacv.cls
% (v1.7, 9 Aug 2023) written by LianTze Lim (liantze@gmail.com), based on the
% CV created by BusinessInsider at http://www.businessinsider.my/a-sample-resume-for-marissa-mayer-2016-7/?r=US&IR=T
%
%% It may be distributed and/or modified under the
%% conditions of the LaTeX Project Public License, either version 1.3
%% of this license or (at your option) any later version.
%% The latest version of this license is in
%%    http://www.latex-project.org/lppl.txt
%% and version 1.3 or later is part of all distributions of LaTeX
%% version 2003/12/01 or later.
%%%%%%%%%%%%%%%%

%% Use the "normalphoto" option if you want a normal photo instead of cropped to a circle
% \documentclass[10pt,a4paper,normalphoto]{altacv}

\documentclass[10pt,a4paper,ragged2e,withhyper]{altacv}
%% AltaCV uses the fontawesome5 package.
%% See http://texdoc.net/pkg/fontawesome5 for full list of symbols.

% Change the page layout if you need to
\geometry{left=1.25cm,right=1.25cm,top=1.5cm,bottom=1.5cm,columnsep=1.2cm}

% The paracol package lets you typeset columns of text in parallel
\usepackage{paracol}
\usepackage{multicol}
\usepackage{tasks}


% Change the font if you want to, depending on whether
% you're using pdflatex or xelatex/lualatex
% WHEN COMPILING WITH XELATEX PLEASE USE
% xelatex -shell-escape -output-driver="xdvipdfmx -z 0" mmayer.tex
\ifxetexorluatex
  % If using xelatex or lualatex:
  \setmainfont{Lato}
\else
  % If using pdflatex:
  \usepackage[default]{lato}
\fi

% Change the colours if you want to
\definecolor{VividPurple}{HTML}{3E0097}
\definecolor{SlateGrey}{HTML}{2E2E2E}
\definecolor{LightGrey}{HTML}{666666}
% \colorlet{name}{black}
% \colorlet{tagline}{PastelRed}
\colorlet{heading}{VividPurple}
\colorlet{headingrule}{VividPurple}
% \colorlet{subheading}{PastelRed}
\colorlet{accent}{VividPurple}
\colorlet{emphasis}{SlateGrey}
\colorlet{body}{LightGrey}

% Change some fonts, if necessary
\renewcommand{\namefont}{\large\bfseries}
% \renewcommand{\personalinfofont}{\footnotesize}
\renewcommand{\cvsectionfont}{\large\bfseries}
\renewcommand{\cvsubsectionfont}{\small\bfseries}

% Change the bullets for itemize and rating marker
% for \cvskill if you want to
\renewcommand{\cvItemMarker}{{\small\textbullet}}
\renewcommand{\cvRatingMarker}{\faCircle}
% ...and the markers for the date/location for \cvevent
% \renewcommand{\cvDateMarker}{\faCalendar*[regular]}
% \renewcommand{\cvLocationMarker}{\faMapMarker*}


% If your CV/résumé is in a language other than English,
% copy-paste from the PDF or run pdftotext, the location
% and date marker icons for \cvevent will paste as correct
% translations. For example Spanish:
% \renewcommand{\locationname}{Ubicación}
% \renewcommand{\datename}{Fecha}


%% Use (and optionally edit if necessary) this .tex if you
%% want to use an author-year reference style like APA(6)
%% for your publication list
% % When using APA6 if you need more author names to be listed
% because you're e.g. the 12th author, add apamaxprtauth=12
\usepackage[backend=biber,style=apa6,sorting=ydnt]{biblatex}
\defbibheading{pubtype}{\cvsubsection{#1}}
\renewcommand{\bibsetup}{\vspace*{-\baselineskip}}
\AtEveryBibitem{%
  \makebox[\bibhang][l]{\itemmarker}%
  \iffieldundef{doi}{}{\clearfield{url}}%
}
\setlength{\bibitemsep}{0.25\baselineskip}
\setlength{\bibhang}{1.25em}


%% Use (and optionally edit if necessary) this .tex if you
%% want an originally numerical reference style like IEEE
%% for your publication list
\usepackage[backend=biber,style=ieee,sorting=ydnt,defernumbers=true]{biblatex}
%% For removing numbering entirely when using a numeric style
\setlength{\bibhang}{1.25em}
\DeclareFieldFormat{labelnumberwidth}{\makebox[\bibhang][l]{\itemmarker}}
\setlength{\biblabelsep}{0pt}
\defbibheading{pubtype}{\cvsubsection{#1}}
\renewcommand{\bibsetup}{\vspace*{-\baselineskip}}
\AtEveryBibitem{%
  \iffieldundef{doi}{}{\clearfield{url}}%
}


%% sample.bib contains your publications
\addbibresource{sample.bib}

\begin{document}
\name{\Large Sai Sampath Kedari}
\tagline{\small Seeking full-time research opportunities in Dynamics Learning, State Estimation, and Control, with interest in foundational machine learning and robust algorithm design.}
% \tagline{\small Seeking full-time opportunities in the field of Dynamics Learning, Motion Planning, State Estimation, and Control in robotic systems.}
% \tagline{\small Interested in the convergence of Statistical Inference, Monte Carlo and Bayesian Filtering, Convex Optimization for Dynamics Learning, Motion Planning, State Estimation, and Control in robotic systems, built on a strong mathematical foundation.}
% Cropped to square from https://en.wikipedia.org/wiki/Marissa_Mayer#/media/File:Marissa_Mayer_May_2014_(cropped).jpg, CC-BY 2.0
%% You can add multiple photos on the left or right
% \photoR{2.5cm}{mmayer-wikipedia-cc-by-2_0}
% \photoL{2cm}{Yacht_High,Suitcase_High}
\personalinfo{
  % Not all of these are required!
  % You can add your own with \printinfo{symbol}{detail}
  \email{sampath@umich.edu}
  \phone{734-510-1994}
  % \mailaddress{Address, Street, 00000 County}
  % \location{Sunnyvale, CA}
  % \homepage{SaiSampathKedari}
  % \twitter{@marissamayer}
  \linkedin{sai-sampath-kedari}
  \github{SaiSampathKedari} % I'm just making this up though.
  \twitter{@SSampathKedari}
%   \orcid{0000-0000-0000-0000} % Obviously making this up too.
  %% You can add your own arbitrary detail with
  %% \printinfo{symbol}{detail}[optional hyperlink prefix]
  % \printinfo{\faPaw}{Hey ho!}
  %% Or you can declare your own field with
  %% \NewInfoFiled{fieldname}{symbol}[optional hyperlink prefix] and use it:
  % \NewInfoField{gitlab}{\faGitlab}[https://gitlab.com/]
  % \gitlab{your_id}
	%%
  %% For services and platforms like Mastodon where there isn't a
  %% straightforward relation between the user ID/nickname and the hyperlink,
  %% you can use \printinfo directly e.g.
  % \printinfo{\faMastodon}{@username@instace}[https://instance.url/@username]
  %% But if you absolutely want to create new dedicated info fields for
  %% such platforms, then use \NewInfoField* with a star:
  % \NewInfoField*{mastodon}{\faMastodon}
  %% then you can use \mastodon, with TWO arguments where the 2nd argument is
  %% the full hyperlink.
  % \mastodon{@username@instance}{https://instance.url/@username}
}



\makecvheader

%% Depending on your tastes, you may want to make fonts of itemize environments slightly smaller
\AtBeginEnvironment{itemize}{\small}

% \cvsection{Objective}
% \begin{itemize}
%     \item Seeking full-time opportunities in the realm of state estimation, dynamics, and controls specifically tailored for mobile robots
% \end{itemize}
    
% \item Seeking Full-time opportunities in the field of Controls, Machine Learning for Mobile robots

%% Set the left/right column width ratio to 6:4.
\columnratio{0.6}
  
% Start a 2-column paracol. Both the left and right columns will automatically
% break across pages if things get too long.
\begin{paracol}{2}

\switchcolumn
\cvsection{EDUCATION}
\cvevent{M.S. in Mechanical Eng. (Robotics)}{University of Michigan, Ann Arbor}{Jan'23 - Apr'24}{GPA: 3.66/4}
\vspace{-0.5em}
\divider
\cvevent{M.S. in Automotive Eng. (Controls \& Dynamics)}{University of Michigan, Ann Arbor}{Aug'21 - Dec'22}{GPA: 3.64/4}
\vspace{-0.5em}
\divider
\cvevent{B.Tech in Mechanical Engineering }{National Institute of Technology Rourkela, India}{Jul'15 - May'19}{GPA: 8.22/10}
\vspace{-0.5em}
    % \resumeSubheading
    %   {National Institute of Technology Rourkela, India}{July 2015 - May 2019}
    %   {Bachelor of Technology in Mechanical Engineering}{ GPA: 8.22/10}

\cvsection{COURSEWORK}
\begin{itemize}[itemsep=-2pt, parsep=4pt]
    \item Inference, Estimation, and Learning
    \item Machine Learning
    \item Computational Data Science \& ML
    \item Intro To Statistical Theory
    \item Stochastic Process II
    \item Probability Distribution Theory
    \item Nonlinear Programming
    \item Nonlinear Systems \& Control 
    \item Linear Systems Theory
    \item Linear Feedback Control
    \item Control Systems Analysis \& Design
    \item Deep Learning
\end{itemize}
% \vspace*{-1.0\multicolsep}

% \vspace{-2em}
% \begin{multicols}{2}
%     % \noindent
%     % \raggedcolumns
%     \begin{itemize}[itemsep=-2pt, parsep=4pt]
%         \item Infer/Est/Learn
%         \item Prob Dist Theory
%         \item Convex Optimizat
%         \item Machine Learning
%         \item Nonlinear Control 
%         \item Math for Robotics
%         \item Lin Sys Theory
%         \item Data Str \& Algo
%     \end{itemize}
% \end{multicols}
% \vspace*{-1.0\multicolsep}

% \cvsection{COURSEWORK}
% \vspace{-0.5em}
% \begin{tabular}{@{}ll}
% Infer/Est/Learn             & Convex Optimization \\
% Probability & Distribution Theory & Machine Learning \\
% Nonlinear Control           & Math for Robotics \\
% Linear Systems Theory       & Data Structures \& Algorithms \\
% \end{tabular}



\cvsection{TECHNICAL SKILLS}

\cvtag{C/C++}
\cvtag{Python}
\cvtag{Matlab}
\cvtag{CMake}\\
\cvtag{TCL/TK Scripting}
\cvtag{Bash Scripting}
\cvtag{ROS1}

 % \begin{itemize}[leftmargin=0.15in, label={}]
 %    \small{\item{
 %     \textbf{Programming Languages}{: C/C++, CMake, OOPs, Python, ROS1, Matlab, TCL/TK Scripting, Batch Scripting}\\
 %     \textbf{Certificates}{: SAE International, ASME-NIT Rourkela, Formula student(FSAE)}
 %    }}
 % \end{itemize}


\switchcolumn
\cvsection{RESEARCH and WORK EXPERIENCE}

\cvevent{Research Assistant}{Prof. Vasileios Tzoumas' Lab, University of Michigan, Ann Arbor}{Aug'24 -- Present}{Ann Arbor, MI}
\begin{itemize}
\item Developing algorithms for learning-based control of agile quadrotors using Koopman operator theory to model aerodynamic effects and enhance data-driven controllers.
\end{itemize}
\vspace{-0.5em}
\divider
\cvevent{Research Study}{ROAHM LAB: Prof Ram Vasudevan,University of Michigan, Ann Arbor}{May'22 -- Aug'22}{Ann Arbor, MI}
\begin{itemize}
\item Performed system identification of Fetch Robot dynamic parameters (link friction, mass, inertia) using open-loop testing, regression, and phase-plane analysis.
\end{itemize}
\vspace{-0.5em}
\divider

\cvevent{Teaching Assistant, Physics 151/241/360- Elec \& Magn, Spec Relativity}{University of Michigan, Ann Arbor}{Aug'22 -- May'24}{Ann Arbor, MI}
\vspace{-0.5em}
\divider

\cvevent{CATIA R\&D Software Developer}{Dassault Systems}{Sep'20 -- Aug'21}{Pune, India}
\begin{itemize}
\item Developed Functional Tolerance \& Annotations Workbench in CATIA using advanced C++, integrating the latest ISO standards.
\end{itemize}
\vspace{-0.5em}
\divider
 
\cvevent{Software Developer}{Altair Engineering}{Sep'2019 -- Sep'20}{Bangalore, India}
\begin{itemize}
\item Developed C++ HyperMesh APIs and MotionView tools for force visualization and two-wheeler dynamics modeling with stability analysis using MDL.
\end{itemize}

\end{paracol}

\cvsection{PROJECTS}
\cvevent{}{AEROSP 567: Statistical Inference, Estimation, and Learning}{Jan'2024 - May'2024}{University of Michigan, Ann Arbor}
\begin{itemize}
    \item Designed Monte Carlo sampling for estimating rare event probabilities in random walks, using variance reduction methods like importance sampling and multilevel Monte Carlo, with the addition of control variates for faster convergence. \href{https://drive.google.com/file/d/1Zj1cKTC8zAlYv7HGjWjmpgrZB7GvdGne/view?usp=sharing}{\textbf{(Project Report)}}
    
    \item Implemented Gaussian Process for scent distribution modeling in a 2D field from sparse sensor data of a flying robot. Developed a search strategy using Bayesian optimization to enhance search efficiency and minimize false positives. \href{https://drive.google.com/file/d/1rmVl7ab2_pvmfXvd2gHSURzPaVzqsEAP/view?usp=sharing}{\textbf{(Project Report)}}
    
    \item Applied Metropolis-Hastings, Adaptive Metropolis, and Delayed Rejection Adaptive Metropolis (DRAM-MCMC) sampling algorithms for parameter inference in dynamical systems, facilitating posterior sampling of parameter distributions. \href{https://drive.google.com/file/d/1U3EeO3B7XSDwNVYHuw6a2UAFkQemIKku/view?usp=sharing}{\textbf{(Project Report)}}
    
    \item Developed Extended Kalman Filter (EKF), Unscented Kalman Filter (UKF), Gauss-Hermite Kalman Filter (GHKF), and Particle Filtering algorithms for state estimation and prediction in nonlinear dynamic systems.\href{https://drive.google.com/file/d/1JPkWKbcHBsgtdJ1h-xOj_W0QmosTLNj2/view?usp=sharing}{\textbf{(Project Report)}}
\end{itemize}

\divider

\cvevent{}{Literature Review : Prof. Alex Gorodetsky \href{https://drive.google.com/file/d/1RkUaKzeyvz4vG0zyT5cfK_vOzAU-4TTz/view?usp=sharing}{\textbf{(Project Report)}} \href{https://drive.google.com/file/d/1XukVD8PcuhjwbzoHYmLOx7Em8Y0Q46XF/view?usp=sharing}{\textbf{(PPT)}}}{Jan'2024 - May'2024}{University of Michigan, Ann Arbor}
\begin{itemize}
    \item Literature Review on "Bayesian system ID: optimal management of parameter model, and measurement uncertainty"
    \item This paper discusses the development of a mathematical objective function for joint parameter-state estimation in dynamic systems, which is fundamental for learning parameters and predicting states in System Identification, and highlights its impact on popular methods like DMD and SINDy

\end{itemize}

% \divider

% \cvevent{Open Source Flight Controller and Airsim Integration for Aerial Manipulation \href{https://github.com/SaiSampathKedari/CP_SocialLSTM.git}{ (Github)}}{Directed Study: Prof. Brent Gillespie}{Jan'2024 - Ongoing}{University of Michigan, Ann Arbor}
% \begin{itemize}
%     \item Facilitating ongoing communication (utilizing MavLink) between our open-source flight controller (RC-pilot, developed in the A2Sys Lab) and Microsoft Airsim \& Unreal Engine to evaluate diverse control algorithms.
% \end{itemize}

\divider

\cvevent{Implementation of "Safe planning in Dynamic Environments using Conformal prediction" research paper\href{https://github.com/SaiSampathKedari/CP_SocialLSTM.git}{ (Github)} }{Research Study: Prof. Dimitra Panagou}{Jan'2023 - May'2023}{University of Michigan, Ann Arbor}
\begin{itemize}
    \item Implemented a Social-LSTM neural network to forecast the motion of surrounding agents around drones, and applied conformal prediction to generate confidence intervals used for optimizing robot motion planning.
\end{itemize}

\divider

\cvevent{Re-implemented an ICML 2022 paper on physics-aware neural networks for PDE modeling.}{Machine Learning Course Project}{Aug'2022 - Dec'2022}{University of Michigan, Ann Arbor}
\begin{itemize}
    \item Implemented physics-aware neural networks, including TCN, ConvLSTM, PhyDNet, and FINN, for the learning of PDEs and ODEs.
\end{itemize}

\cvsection{MATH FOUNDATIONS}
\textit{I believe that mastering and developing an intuitive understanding of mathematical foundations is the key to unlocking new possibilities in research and innovation. The subjects below reflect this commitment—each studied rigorously with all chapter-end exercises solved.}
\vspace{1em}
\begin{itemize}
    \item \textbf{Real Analysis:} Elementary Analysis by Kenneth Ross -  \href{https://github.com/SaiSampathKedari/Real-Analysis}{\textbf{(Github)}}
    
    \item \textbf{Probability and Distribution Theory:} Casella \& Berger - 
    \href{https://github.com/SaiSampathKedari/Probability-and-Distribution-Theory}{\textbf{(Github)}}
    
    \item \textbf{Statistical Inference Theory:} Casella \& Berger     \href{https://github.com/SaiSampathKedari/Statistical-Inference-Theory}{\textbf{(Github)}}
    
    \item \textbf{Matrix Methods for Machine Learning:}     \href{https://github.com/SaiSampathKedari/Matrix-Methods-Linear-Algebra-Machine-Learning}{\textbf{(Github)}}

    \item \textbf{Convex-Optimization:} Stephen Boyd \& Lieven Vandenberghe     \href{https://github.com/SaiSampathKedari/UMich-IOE611-Convex-Optimization}{\textbf{(Github)}}

    \item \textbf{The Fourier Transform \& its Applications:} - Stanford EE261     \href{https://github.com/SaiSampathKedari/Stanford-EE261-The-Fourier-Transform-and-its-Applications}{\textbf{(Github)}}

    \item \textbf{Signals \& Systems:} Alan V. Oppenheim, Willsky, Nawab     \href{https://github.com/SaiSampathKedari/Signals-and-Systems}{\textbf{(Github)}}

    \item \textbf{Differential Equations:} MIT 18.03 \& Edwards-Penney     \href{https://github.com/SaiSampathKedari/Differential-Equations}{\textbf{(Github)}}
\end{itemize}

\vfill
\begin{center}
\scriptsize Last updated: \today \quad • \quad Sai Sampath Kedari
\end{center}

\end{document}
